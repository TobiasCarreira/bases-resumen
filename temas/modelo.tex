\chapter{Modelo}

\section{Introducción}

Profundizar en \cite{molina2009}, capítulo 1.

System Catalog: parte del RDBMS que contiene la metadata, como el esquema, indices, triggers, configuraciones, etc.

\begin{description}
	\item[DML (Data Manipulation Language):] Es el usado en los DBMS para modificar las instancias: obtener,
agregar, cambiar, etc. Algunas de sus funciones en SQL son SELECT, INSERT, DELETE.
	\item[DDL (Data Definition Language):] Se usa en los DBMS para alterar el esquema de la base: crear, agregar
atributo, renombrar. Algunas de sus funciones en SQL son CREATE, ALTER, DROP, RENAME.
\end{description}

\subsection{Niveles de abstracción}

Tenemos tres niveles de abstracción:

\begin{description}
	\item[Nivel físico:] archivos, indices y estructuras de almacenamiento para poder utilizar la base de forma eficiente
	\item[Modelo conceptual:] abstracción del mundo real que la base de datos intenta representar, utilizando conceptos lógicos
	\item[Vistas:] sub-esquema del modelo conceptual, o abstracción de este
\end{description}

Separar estos tres niveles nos proveen dos independencias relevantes:

\begin{description}
	\item[Independencia física:] capacidad del sistema de ejecutar cambios sobre la definición y estructura de los datos sin que esto afecte la base de datos conceptual
	\item[Independencia lógica:] la capacidad del sistema de
cambiar el esquema conceptual, sin cambiar la vista
lógica que el usuario tiene de los datos. Más difícil de conseguir
\end{description}

\subsection{Modelo Entidad Relación}

Modelo Entidad Relación (MER) es un modelo conceptual de alto nivel, representando entidades del mundo real y sus relaciones. El Diagrama Entidad Relación (DER) es la herramienta gráfica que permite visualizar las entidades del modelo y sus relaciones.

\begin{description}
	\item[Entidad:] cosa del mundo real que tiene una existencia independiente. Se agrupan en clases o tipos de entidades.
	\item[Relación:] asociación entre entidades. Tienen una aridad (unaria, binaria, ternaria) y cardinalidad (uno a uno, uno a muchos, muchos a muchos). Pueden ser parciales (opcionales) o totales. Es débil si no tiene existencia independiente.
	\item[Atributo:] propiedad perteneciente a una entidad o a una relación muchos a muchos. Puede ser simple, compuesto o derivado (aunque solo los simples terminan modelándose de forma directa). Pueden ser multivaluados (tener varios valores a la vez)
	\item[Llaves:] grupo de atributos que identifica a una entidad. Identifican unívocamente a cada individuo de una clase.
\end{description}

El MER puede mapearse a tablas en el Modelo Relacional.

\subsection{Modelo Relacional}

Los registros se abstraen en \textbf{relaciones} (tablas). Cada registro en una relación se denomina \textbf{tupla} (filas). El orden entre las tuplas no está especificado. La fila cuenta con varios atributos (columnas), y cada uno tiene un dominio.

\begin{description}
	\item[Esquema de base de datos:] colección (no vacía y finita) de esquemas de relación
	\item[Esquema de relación:] consiste de un nombre y un conjunto finito de atributos (columnas).
\end{description}

\section{Lenguajes de consulta}

Profundizar en \cite{elmasri2015}, capítulo 8 (o capítulo 6 de la 6ta edición).

\subsection{Álgebra Relacional}

\begin{description}
	\item[Álgebra:] dominio + operaciones de aridad finita + axiomas
\end{description}

El Álgebra Relacional es un lenguaje formal para consultas sobre un modelo relacional. Es la base para pensar optimizaciones de consultas. Es \textbf{procedural}. Las operaciones toman relaciones y devuelven relaciones.

Operadores:

\begin{description}
	\item[SELECT] $\sigma_{<condicion>}(R)$ filtra (partición horizontal). Es un operador unario. Es conmutativa y se pueden simplificar dos SELECTs uniendo las condiciones con un AND (cascada)
	\item[PROJECT] $\pi_{<lista de atributos>}(R)$ se queda con un subconjunto de columnas (partición vertical). Operación unaria. \textbf{Remueve duplicados}
	\item[RENAME] $\rho(S, R)$ o $\rho(S(A_1 \to B_1, ..., A_n \to B_n), R)$ renombra una relación o sus columnas
	\item[Operaciones de conjuntos] unión $R \cup S$, intersección $R \cap S$, diferencia $R - S$. Deben ser unión compatibles (las columnas matchean en tipo)
	\item[PRODUCTO CARTESIANO] $R \times S$ combina cada tupla de una relación
con cada una de las tuplas de la otra relación
	\item[JOIN] $R \bowtie_{<condicion>} S$ producto cartesiano + filtro por condición con columnas de ambas relaciones. Es conmutativa y la cantidad de columnas del resultado es la suma de la cantidad de cada tabla
	\item[NATURAL JOIN] unifica dos relaciones usando las columnas que comparten nombre
	\item[OUTER JOIN] conserva tuplas que están en alguna de las relaciones pero no matchean con ninguna de la otra relación (completando atributos faltantes con NULL)
	\item[DIVISION] $R \div S$ retorna las tuplas de R que se encuentran emparejadas con TODAS las tuplas de S. Requiere que $atributos(S) \subseteq atributos(R)$. Resultado contiene los atributos de R no compartidos con S. Se puede expresar con otras operaciones, no suele implementarse en SQL
\end{description}

\subsection{Cálculo Relacional de Tuplas}

A diferencia de AR, CRT es declarativo: el lenguaje permite expresar cómo es el conjunto de respuestas que
buscamos, las caracteristicas de las tuplas que deben estar en la respuesta. SQL se basa en CRT.

Las consultas se escriben en la forma $\{t | COND(t)\}$ donde $t$ es de tipo tupla y $COND$ es un predicado (fórmula bien formada) cuya única variable libre debe ser $t$. Las fórmulas bien formadas permiten:

\begin{itemize}
	\item Ver pertenencia a una relación
	\item Comparar atributos de relaciones con literales u otros atributos de relaciones
	\item Operaciones básicas de lógica proposicional y de primer orden
\end{itemize}

\subsection{Expresividad de los Lenguajes de Consulta}

\begin{description}
	\item[Expresividad] amplitud de ideas que pueden ser representadas y comunicados en un lenguaje. En un lenguaje de consultas, equivale al conjunto de consultas que nos permite hacer
\end{description}

CRT es una especialización de la Lógica de Primer Orden aplicada a bases de datos. No requiere teoría de modelos, ya que sólo interesa saber la validez de una fórmula en un modelo puntual: el esquema de la base de datos.

CRT es equivalente a AR si nos restringimos a las \textbf{expresiones seguras}. Estas son las que garantizan producir una \textbf{cantidad finita de tuplas como resultado}. Una forma alternativa de verlo es que los valores en el resultado son parte del dominio de la expresión (son constantes en la consulta, o aparecen en alguna tupla de alguna de las relaciones referenciadas en esta).

No todo es expresable en estos lenguajes: algunos requieren recursión/iteración (ejemplo, camino de largo desconocido en un grafo). No se pueden expresar sentencias en LPO con n cuantificadores anidados para cualquier n natural. Existen herramientas matemáticas que permiten demostrar
que ciertas propiedades (consultas) no son expresables en
LPO: \textbf{Teorema Ehrenfeucht-Fraisse}.

SQL se basa en CRT con expresiones seguras, pero se suele agregar:

\begin{itemize}
	\item Recursión
	\item Agrupamiento y agregación
	\item Operaciones aritméticas
\end{itemize}

\section{Normalización}

Profundizar en \cite{elmasri2015}, capítulo 15 (hasta 15.5 inclusive).

\subsection{Pautas de diseño}

\begin{description}
	\item[Semantica clara:] que se entienda, división de responsabilidades
	\item[Reducir información redundante:] que no se repita la misma información en varios lugares, puede producir anomalías si no se trata apropiadamente
	\item[Reducir nulls:] atributos que muchas veces no aplican a individuos de la entidad (por ejemplo, si se tienen varias tablas en una)
	\item[Evitar tuplas espúreas:] diseñar esquemas de forma que los JOINS se hagan usando claves, para evitar que se generen tuplas sin sentido
\end{description}

¿Qué problemas podría presentar un esquema?

\begin{description}
	\item[Anomalías de inserción:] Cuando se tienen atributos no pertinentes o aglomerados en una relación, es posible que en la inserción algunos de estos deban llevar NULL.
	\item[Anomalías de actualización:] por redundancia, se debe actualizar varias veces el mismo valor en distintos lugares.
	\item[Anomalías de borrado:] puede ocurrir pérdida de información debido a que queremos borrar algo específico, pero el esquema de relación tiene más información.
\end{description}

Todo esto surge de tener información redundante.

\subsection{Formas normales}

Propiedades deseables:

\begin{description}
	\item[Nonadditive Join (Lossless Join):] garantía de que no ocurre problema de
generación de tuplas espúreas. La relación original tiene que poder ser
recuperada de la descomposición.
	\item[Preservación de DF:] garantía de que cada DF se encuentra representada en algún esquema resultante de la descomposición
\end{description}

Claves:

\begin{description}
	\item[Super clave (SK):] subconjunto de los atributos tal que no puede existir dos tuplas con los mismos valores en ellos
	\item[Clave (K):] super clave minimal
	\item[Clave candidata (CK):] cada clave es candidata a ser primaria
	\item[Clave primaria (PK):] una de las CK, elegida arbitrariamente
	\item[Clave secundaria:] CK que no es PK
	\item[Atributo primo:] es parte de alguna clave candidata
\end{description}

Tipos de dependencias funcionales (DFs):

\begin{description}
	\item[Completa:] $X \to Y$ es completa si al eliminar un atributo de X, la DF deja de cumplirse
	\item[Parcial:] si no es completa
	\item[Transitiva:] Una DF $X \to Y$ es transitiva si tenemos un conjunto de atributos no primos Z tal que $X \to Z$ y $Z \to Y$
	\item[Cubrimiento minimal:] conjunto de DFs tal que del lado derecho de las DFs solo hay un atributo, no se pueden sacar nada del lado izquierdo, no se puede sacar ninguna DF sin dejar de ser equivalente (no hay redundancia)
\end{description}

Formas normales:

\begin{description}
	\item[1FN:] solo debe tener atributos simples
	\item[2FN:] \textbf{todo} atributo no primo de la relación \textbf{depende de manera completa de la PK}
	\item[3FN:] \textbf{ningún} atributo no primo \textbf{depende transitivamente de la PK}
	\item[BCFN (Boyce-Codd):] para toda dependencia funcional no trivial, el lado izquierdo es SK. Descomponer para cumplirla puede llevar a pérdida de DFs
\end{description}

\subsection{Inferencia}

Clausura de un conjunto de DFs F: todas las DFs que pueden inferirse a partir de F con los 3 axiomas de Amstrong (es un procedimiento sano y completo):

\begin{description}
	\item[Regla reflexiva:] $Y \subseteq X$ entonces $X \to Y$
	\item[Regla de incremento:] $X \to Y$ entonces $XZ \to YZ$
	\item[Regla transitiva:] $X \to Y$ y $Y \to Z$ entonces $X \to Z$
\end{description}

Dos conjuntos de DFs son equivalentes si sus clausuras son iguales.

\section{Datos semi estructurados - XML}

Los datos no residen en campos fijos o registros o tuplas,
pero contienen elementos (marcas) que pueden separar los
datos de manera jerárquica.
La información sobre la estructura esta mezclada con los
datos.

Un ejemplo es XML:

\begin{itemize}
	\item Archivos en texto plano
	\item Los documentos tienen “tags” que le dan información extra sobre las secciones del documento
	\item Datos estructurados en forma de árbol (solo una raíz) y permite anidamiento: no cumple 1FN pero es conveniente para transmisión de datos
	\item Provee una interfaz robusta con herramientas estables de parsing
	\item Puede contener una descripción de su gramática
	\item Suele tener redundancia
\end{itemize}

Ejemplo: \texttt{<price currency="euro">547.89</price>}, donde \texttt{price} es el nombre del elemento/tag, \texttt{currency} un atributo, \texttt{euro} el valor del atributo y \texttt{547.89} el contenido del elemento

Restricciones para que esté bien formado:

\begin{itemize}
	\item Los tags tienen que matchear y anidarse apropiadamente
	\item Los atributos son únicos a nivel del tag
\end{itemize}

\textbf{Document Type Definition} (DTD) provee un esquema para un XML.
Son gramáticas libres de contexto.
\textbf{XML Schema} es un lenguaje de esquema más sofisticado.
Soporta tipos, restricciones, tipos complejos, llaves.

Lenguajes estándar para consultas:

\begin{description}
	\item[XPath:] lenguaje simple para definir expresiones de caminos
	\item[XSLT:] lenguaje simple diseñado para traducir de XML a XML y de XML a HTML.
	\item[XQuery:] un lenguaje rico de consulta para XML
	\item[SQL/XML:] extensión de SQL para hacer consultas a XML
\end{description}
