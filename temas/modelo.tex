\chapter{Modelo}

\section{Introducción}

Profundizar en \cite{molina2009}, capítulo 1.

\subsection{Niveles de abstracción}

Tenemos tres niveles de abstracción:

\begin{description}
	\item[Nivel físico] archivos, indices y estructuras de almacenamiento para poder utilizar la base de forma eficiente
	\item[Modelo conceptual] abstracción del mundo real que la base de datos intenta representar, utilizando conceptos lógicos
	\item[Vistas] sub-esquema del modelo conceptual, o abstracción de este
\end{description}

Separar estos tres niveles nos proveen dos independencias relevantes:

\begin{description}
	\item[Independencia física] capacidad del sistema de ejecutar cambios sobre la definición y estructura de los datos sin que esto afecte la base de datos conceptual
	\item[Independencia lógica] la capacidad del sistema de
cambiar el esquema conceptual, sin cambiar la vista
lógica que el usuario tiene de los datos. Más difícil de conseguir
\end{description}

\subsection{Modelo Entidad Relación}

Modelo Entidad Relación (MER) es un modelo conceptual de alto nivel, representando entidades del mundo real y sus relaciones. El Diagrama Entidad Relación (DER) es la herramienta
gráfica que permite visualizar las entidades del modelo y
sus relaciones.

\begin{description}
	\item[Entidad] cosa del mundo real que tiene una existencia independiente. Se agrupan en clases o tipos de entidades.
	\item[Relación] asociación entre entidades. Tienen una aridad (unaria, binaria, ternaria) y cardinalidad (uno a uno, uno a muchos, muchos a muchos). Pueden ser parciales (opcionales) o totales.
	\item[Atributo] propiedad perteneciente a una entidad o a una relación muchos a muchos. Puede ser simple, compuesto o derivado (aunque solo los simples terminan modelándose de forma directa).
	\item[Llaves] grupo de atributos que identifica a una entidad. Identifican unívocamente a cada individuo de una clase.
\end{description}

El MER puede mapearse a tablas en el Modelo Relacional.

\subsection{Modelo Relacional}

Los registros se abstraen en \textbf{relaciones} (tablas). Cada registro en una relación se denomina \textbf{tupla} (filas). El orden entre las tuplas no está especificado. La fila cuenta con varios atributos (columnas), y cada uno tiene un dominio.

\begin{description}
	\item[Esquema de base de datos] colección (no vacía y finita) de esquemas de relación
	\item[Esquema de relación] consiste de un nombre y un conjunto finito de atributos (columnas).
\end{description}

\section{Lenguajes de consulta}

Profundizar en \cite{elmasri2015}, capítulo 8 (o capítulo 6 de la 6ta edición).

\subsection{Álgebra Relacional}

\begin{description}
	\item[Álgebra] dominio + operaciones de aridad finita + axiomas
\end{description}

El Álgebra Relacional es un lenguaje formal para consultas sobre un modelo relacional. Es la base para pensar optimizaciones de consultas. Es \textbf{procedural}. Las operaciones toman relaciones y devuelven relaciones.

Operadores:

\begin{description}
	\item[SELECT] $\sigma_{<condicion>}(R)$ filtra (partición horizontal). Es un operador unario. Es conmutativa y se pueden simplificar dos SELECTs uniendo las condiciones con un AND (cascada)
	\item[PROJECT] $\pi_{<lista de atributos>}(R)$ se queda con un subconjunto de columnas (partición vertical). Operación unaria. \textbf{Remueve duplicados}
	\item[RENAME] $\rho(S, R)$ o $\rho(S(A_1 \to B_1, ..., A_n \to B_n), R)$ renombra una relación o sus columnas
	\item[Operaciones de conjuntos] unión $R \cup S$, intersección $R \cap S$, diferencia $R - S$. Deben ser unión compatibles (las columnas matchean en tipo)
	\item[PRODUCTO CARTESIANO] $R \times S$ combina cada tupla de una relación
con cada una de las tuplas de la otra relación
	\item[JOIN] $R \bowtie_{<condicion>} S$ producto cartesiano + filtro por condición con columnas de ambas relaciones
	\item[NATURAL JOIN] unifica dos relaciones usando las columnas que comparten nombre
	\item[OUTER JOIN] conserva tuplas que están en alguna de las relaciones pero no matchean con ninguna de la otra relación (completando atributos faltantes con NULL)
	\item[DIVISION] $R \div S$ retorna las tuplas de R que se encuentran emparejadas con TODAS las tuplas de S. Requiere que $atributos(S) \subseteq atributos(R)$. Resultado contiene los atributos de R no compartidos con S. Se puede expresar con otras operaciones, no suele implementarse en SQL
\end{description}

\subsection{Cálculo Relacional de Tuplas}

A diferencia de AR, CRT es declarativo: el lenguaje permite expresar cómo es el conjunto de respuestas que
buscamos, las caracteristicas de las tuplas que deben estar en la respuesta. SQL se basa en CRT.

Las consultas se escriben en la forma $\{t | COND(t)\}$ donde $t$ es de tipo tupla y $COND$ es un predicado (fórmula bien formada) cuya única variable libre debe ser $t$. Las fórmulas bien formadas permiten:

\begin{itemize}
	\item Ver pertenencia a una relación
	\item Comparar atributos de relaciones con literales u otros atributos de relaciones
	\item Operaciones básicas de lógica proposicional y de primer orden
\end{itemize}

\subsection{Expresividad de los Lenguajes de Consulta}

\begin{description}
	\item[Expresividad] amplitud de ideas que pueden ser representadas y comunicados en un lenguaje. En un lenguaje de consultas, equivale al conjunto de consultas que nos permite hacer
\end{description}

CRT es una especialización de la Lógica de Primer Orden aplicada a bases de datos. No requiere teoría de modelos, ya que sólo interesa saber la validez de una fórmula en un modelo puntual: el esquema de la base de datos.

CRT es equivalente a AR si nos restringimos a las \textbf{expresiones seguras}. Estas son las que garantizan producir una \textbf{cantidad finita de tuplas como resultado}. Una forma alternativa de verlo es que los valores en el resultado son parte del dominio de la expresión (son constantes en la consulta, o aparecen en alguna tupla de alguna de las relaciones referenciadas en esta).

No todo es expresable en estos lenguajes: algunos requieren recursión/iteración (ejemplo, camino de largo desconocido en un grafo). No se pueden expresar sentencias en LPO con n cuantificadores anidados para cualquier n natural. Existen herramientas matemáticas que permiten demostrar
que ciertas propiedades (consultas) no son expresables en
LPO: \textbf{Teorema Ehrenfeucht-Fraisse}.

SQL se basa en CRT con expresiones seguras, pero se suele agregar:

\begin{itemize}
	\item Recursión
	\item Agrupamiento y agregación
	\item Operaciones aritméticas
\end{itemize}

\section{Normalización}

Profundizar en \cite{elmasri2015}, capítulo 15 (hasta 15.5 inclusive).

\subsection{Pautas de diseño}

\subsection{Formas normales}

\section{Datos semi estructurados - XML}

\section{Datos abiertos}
