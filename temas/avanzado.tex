\chapter{Temas avanzados}

\section{InMemory DB}

Una “in-memory database” es una base de datos que carga TODA la base de datos en memoria.
Puede tener persistencia.
Si se quiere garantizar persistencia, se suele escribir constantemente al log, ya que es más rápido (es un append).
A veces almacenan los datos por columnas en vez de por filas.

\section{ORM}

Algunas extensiones:

tipos definidos por el usuario

referencias a atributos de otras tablas (objetos)

herencia

\section{Interoperabilidad}

\subsection{Data Warehouse}

Gran base de datos preprocesados con una finalidad. Es un conjunto de datos transaccionales de un tema específico, puede integrar distintas bases de datos. Se organiza para facilitar las consultas y el análisis. Puede incluir valores sumarizados y derivados.

\begin{description}
	\item[OLTP:] online transaction process, datos operacionales, preparadas para tareas cortas, permite transacciones eficientes
	\item[OLAP:] online analytical process, preparadas para grandes consultas. Se organiza por tema
\end{description}

Metadata: debe documentarse toda la información de generación del DW. Fuentes de datos, transformaciones, reglas de limpieza, etc

ETL y ELT son procedimientos para armas DW a partir de fuentes operacionales.

\begin{description}
	\item[Extracción (migración + limpieza):] se copian los datos a un área de trabajo y se limpian (manejo de outliers, de missings, estandarización)
	\item[Transformación:] adapta los datos al modelo lógico del DW mediante una serie de reglas. Suele des-normalizarse/agregar sumarizaciones aunque debe existir alguna forma de navegar los datos hasta el nivel de detalle (drill down)
	\item[Load:] se cargan los datos transformados en las tablas del DW
\end{description}

ELT replica los datos (hace un espejo) leyendo logs antes de transformarlos.

Para modelar un DW, debe tenerse una idea de las preguntas frecuentes que se le harán a los datos.
Modelo dimensional: se tienen hechos (entidades principales) con varias dimensiones (tiempo, lugar, usuario) y una métrica asignada para cada posible valor de las dimensiones, formando un hipercubo.
Las dimensiones pueden tener jerarquías: el tiempo se divide en años, que se dividen en meses, que se dividen en días.
Operaciones comunes:

\begin{description}
	\item[Roll up:] sumarizar por una dimensión
	\item[Drill down:] desagregar
	\item[Slice:] quedarse con un subconjunto del hipercubo, filtrar por un único valor de una dimensión
	\item[Dice:] como slices, pero puede ser con más dimensiones y más valores por dimensión
\end{description}

Suele usarse almacenamiento por columna ya que uno se interesa en algunas de columnas y no el registro entero

\section{Ontologías}

\section{Calidad y gobierno de datos}

Existen datos en formatos muy diversos: tablas, documentos, grafos, streams (IoT)

\begin{description}
	\item[Calida de datos:] que tan bien reflejan la realidad
\end{description}

Problemas:

\begin{itemize}
	\item Datos erróneos, nulls con múltiples significados, malos defaults, combinaciones de atributos que son posibles pero sospechosos
	\item Datos poco informativos (muchas filas con valor "otro")
	\item Datos desactualizados
	\item Datos no consistentes entre distintas tablas/bases
	\item Distintos formatos no estandarizados (calles, fechas), distintas unidades
	\item Datos que no cumplen restricciones (fuera de rango, fuera de las opciones validas)
	\item Sesgos en los datos
\end{itemize}

Qué hacer:

\begin{itemize}
	\item análisis estadístico, detección de outliers
	\item no permitir que la base de datos se contamine con datos incorrectos, investigar si corresponde agregarlo
\end{itemize}

Problemas éticos: la información errónea puede afectar la vida de las personas

\begin{description}
	\item[Gobierno de datos:] desarrollo y ejecución de arquitecturas, prácticas y procedimientos que manejan adecuadamente las necesidades del ciclo de vida de los datos. Incluyen calidad, arquitectura, seguridad, metadata. Es tanto responsabilidad de sistemas (arquitectura, infra) como del negocio (dueño de datos, definición de los datos)
\end{description}

\section{Seguridad}

\section{Mineria de datos}
