\chapter{Implemetación}

\section{Transacciones}

Una transacción es un conjunto de instrucciones que forman una unidad lógica de procesamiento.

La base de datos está compuesta por items, que pueden ser registros, atributos, bloques de memoria.
Cada item tiene un identificador único (que no necesita ser conocido por el programador).
Una transacción es una secuencia de lecturas y escrituras de items.

\begin{description}
	\item[read-set:] conjunto de items leídos por una transacción
	\item[write-set:] conjunto de items escritos por una transacción
\end{description}

Problemas comunes:

\begin{description}
	\item[Lost Update:] T1 lee X, T2 lee X, T1 escribe X usando el valor que leyó, T2 escribe X usando el valor que leyó
	\item[Dirty Read:] leo algo que luego es abortado y rollbackeado
	\item[Unrepeatable read:] Ocurre cuando dentro de una transacción, dos reads (uno atrás del otro o sin modificar los datos), leen diferentes valores del mismo dato
	\item[Incorrect Summary:] T1 modifica algo que T2 estaba sumarizando y T2 obtiene mal el resultado
\end{description}

Una Transacción T alcanza su punto de commit cuando: operaciones ejecutadas correctamente y grabadas en el log.

Propiedades ACID:

\begin{description}
	\item[Atomicidad:] una transacción es una unidad atómica. Responsabilidad del subsistema de recovery
	\item[Consistencia:] una transacción mueve la BD de un estado consistente a otro. Responsabilidad dividida entre el programador y el módulo de restricciones de integridad del DBMS
	\item[Aislamiento:] las transacciones no deben interferir entre ellas. Deben correr como si fueran la única transacción en el sistema. Impuesto por el subsistema de concurrencia
	\item[Durabilidad:] los cambios commiteados deben persistir en la BD. Responsabilidad del subsistema de recovery
\end{description}

\section{NoSQL}

\begin{description}
	\item[Basadas en documentos:] cada entrada es un documento, sin esquema pero con algún formato (semi estructurado). Ejemplo: JSONs en MongoDB
\end{description}

Sharding: metodología que distribuye filas de una misma tabla en distintos nodos. Es fragmentación horizontal.

BASE:

\begin{description}
	\item[Basic Availability:] se intenta poder responder consultas lo más posible (sistema altamente distribuido)
	\item[Soft state:] cambios sin necesidad de entradas nuevas
	\item[Eventual consistency]: la base de datos debe converger en algún momento a un estado único
\end{description}

CAP: solo se puede cumplir dos de tres

\begin{description}
	\item[Consistencia:] una única copia de los datos consistente
	\item[Disponibilidad/Availability:] cada solicitud devuelve una respuesta
	\item[Tolerancia a particiones:] el sistema sigue funcionando sin importar la cantidad de mensajes que se pierdan o retrasen en la red
\end{description}

Las bases de datos NoSQL suelen no tener esquema. Existen diferentes tipos.

\subsection{Clave-Valor}

\begin{itemize}
	\item diccionario donde el valor puede ser potencialmente cualquier cosa
	\item muy simples, escalan fácilmente pero no aceptan consultas complejas
	\item items pueden almacenarse en ``dominios'' que son similares a las tablas, con la única diferencia que no tienen un esquema
	\item pueden producir mucha redundancia
	\item no existe integridad referencial (por ejemplo, quizá hay claves guardadas que llevan a items que fueron borrados)
	\item adecuadas para objetos
\end{itemize}

\subsection{Documentos}

\begin{itemize}
	\item diccionario clave-valor, pero ahora el valor es un documento semi-estructurado (XML, JSON), sin tener necesariamente un esquema
	\item pueden generar metadata para indexar
\end{itemize}

Un ejemplo es MongoDB:

\begin{itemize}
	\item permite \textbf{fragmentación horizontal o sharding}: se distribuyen las filas en varios nodos según rangos de la clave
	\item permite operaciones complejas estilo map/reduce/group by/regex
\end{itemize}

\subsection{Familia de columnas}

\begin{description}
	\item[Familia de columnas estándar:] cada item tiene una clave y un conjunto de columnas (clave+valor)
	\item[Familia de columnas super:] cada item tiene una clave y un conjunto de conjuntos de columnas
\end{description}

\begin{itemize}
	\item están pensadas para grandes cantidades de datos
	\item permiten partición vertical
	\item permiten indexación
	\item soporta datos semi-estructurados
	\item no son adecuadas para datos relacionales
	\item no proveen consistencia inmediata
\end{itemize}

\subsection{Grafos}

\begin{itemize}
	\item datos almacenados en nodos
	\item relaciones dirigidas y tipadas en ejes
	\item permite atributos tanto en nodos como en ejes
	\item ejemplo: Neo4j, cumple ACID y soporta lenguajes de consulta especiales
\end{itemize}

\section{Optimización de consultas}

El optimizador recibe una expresión algebraica de la consulta y devuelve un plan de ejecución eficiente.
Puede haber múltiples planes posibles ya que se puede \textbf{reescribir la expresión de formas lógicamente equivalentes} y porque hay expresiones que cuentan con \textbf{múltiples algoritmos posibles}.

Una consulta SQL puede ser expresada en un árbol de expresión en AR donde las hojas son las relaciones y los nodos internos son operaciones.
El árbol canónico es el inicial, no optimizado.

\subsection{Índices}

Estructura adicional que permite el acceso eficiente según algún atributo.

Tipos de índices:

\begin{description}
	\item[Clustered:] los datos del archivo están físicamente en el mismo orden que el índice (solo puede haber uno por tabla)
	\item[Denso:] tiene una entrada de índice por cada valor posible de la clave
	\item[Primario:] tiene todos los registros completos de los archivos (vs tener solo el id de fila)
\end{description}

Organización de archivos:

\begin{description}
	\item[Heap:] colección desordenada de registros, agrupados en bloques
	\item[Sorted:] colección ordenada con respecto a uno de los atributos
	\item[Árbol B+ (de orden d):] es un árbol balanceado, tiene entre d/2 y d claves por nodo interno. Las hojas tienen los datos (o id de fila si son unclustered), es eficiente para consultas por rangos
	\item[Hash estático:] tabla con cantidad fija de buckets. Cada bucket es una lista enlazada. Ideal para búsquedas por igualdad
\end{description}

\subsection{Optimizaciones algebraicas}

\begin{itemize}
	\item Hacer proyecciones y selecciones lo antes posible (mas en las hojas).
	\item Aplicar primero los joins mas selectivos
	\item Cambiar producto cartesiano + seleccion por join
	\item Hacer primero las selecciones más selectivas (Selectividad: proporción de tuplas que cumplen la condición)
	\item Hacer seleccion de relaciones y luego producto cartesiano en vez de producto cartesiano y luego seleccion: $\sigma(R \times S) = \sigma(R) \times \sigma(S)$
	\item VER MAS EN LA TEÓRICA
\end{itemize}

\subsection{Algoritmos de Join}

\begin{description}
	\item[Block Nested Loops Join (BNLJ):] dos loops (uno por cada relación a juntar) anidados y un if para la condición
	\item[Index Nested Loops Join (INLJ):] dos loops anidados, pero ahora en vez de hacer if, el loop interno se hace sobre el resultado de la búsqueda en un índice correspondiente al atributo de la junta
	\item[Sorted Merge Join (SMJ):] se ordenan ambas tablas (o ya están ordenadas por un indice) y se mergea como en merge sort. El resultado queda ordenado por el atributo de la junta
\end{description}

\section{Schedules}

Profundizar en capítulo 20 de \cite{elmasri2015}.

Una historia completa H sobre un conjunto de transacciones T define el orden parcial en que se ejecutan las operaciones de las transacciones.
Dos operaciones entran en conflicto en conflicto si

\begin{itemize}
	\item pertenecen a distintas transacciones
	\item operan sobre un mismo item
	\item al menos una es de escritura
\end{itemize}

Una historia debe respectar el orden parcial de las operaciones dentro de cada transacción y ,para todo par de operaciones en conflicto, debe asignarles un orden.

\textbf{Conflicto equivalencia:} dos historias $H_1$ y $H_2$ son conflicto equivalentes si tienen las mismas transacciones y el orden de las operaciones en conflicto de las transacciones no abortadas es el mismo.

\textbf{Una historia $T_i$ lee de otra $T_j$ ($i <> j$)} si $T_i$ lee algo escrito por $T_j$ (y $T_j$ no hizo abort, al menos por ahora).

Tipos de Historias:

\begin{description}
	\item[Serializable:] equivalente a una historia serial. Para chequear se hace digrafo de precedencia (operaciones en conflicto entre transacciones) y no tiene que tener ciclos
	\item[Recuperable (RC):] si $i <> j$ entonces $c_j < c_i$
	\item[Evita aborts en cascada (ACA):] si $i <> j$, $T_j$ ya había commiteado antes de esa lectura, es decir, se leen solo datos ya commiteados. De esa forma se pueden evitar los dirty reads sin abortar como en RC
	\item[Estricta:] no lee ni escribe un item que fue escrito por una transacción que no commiteó
\end{description}

\section{Concurrencia}

Profundizar en capítulo 21 de \cite{elmasri2015}.

Problemas:

Lost Update
Phantom Read: hago dos selects iguales en la misma transaccion y dan conjuntos de tuplas distintos

Locks

Timestamping

Timestamping con multiversion: busca evitar \textbf{read too late}, al escribir no se pisa el registro sino que se crea una nueva versión y se le asigna un timestamp. Asi, si una transacción más vieja quiere leer un registro que fue escrito por una transacción mas nueva, puede leer la versión correspondiente. No evita aborts en cascada ni write too late.

Niveles Aislamiento en SQL (cada nivel es más estricto que los anteriores):

\begin{description}
	\item[READ UNCOMMITED:] previene Lost Updates
	\item[READ COMMITED:] previene Dirty Reads
	\item[REPEATABLE READS:] previene Non Repeatable Reads
	\item[SERIALIZABLE:] previene Phantom Reads
\end{description}

\section{Recuperación}

\subsection{Política Undo}

Por cada item a escribir, primero se registra en el log y luego se escribe en disco.
Se agrega COMMIT al log luego de escribit todos los cambios a disco.
Para recuperarse, se recorre el log desde el principio y se van deshaciendo las transacciones incompletas (no commiteadas ni abortadas).

\subsection{Política Redo}

Se registran los cambios y el COMMIT en el log, luego se bajan los cambios a disco.
Para recuperarse, se rehacen las transacciones commiteadas.
Como no se bajan los cambios a disco hasta luego del commit, requiere más buffer en memoria.

\subsection{Política Undo/Redo}

Las filas del log tienen las escrituras con valor viejo y valor nuevo para poder hacer redo.
Se escribe en el log antes de bajar el cambio a disco.
Para recuperarse, aplica Redo a las commiteadas y luego Undo a las incompletas.

\subsection{Checkpoints}

\begin{description}
	\item[Checkpoint quiescente:] no permite nuevas transacciones hasta que terminen las que estaban pendientes de commitear
	\item[Checkpoint no quiescente:] permite nuevas transacciones, solo espera a las que estaban pendientes de antes para terminar el checkpoint
\end{description}

\section{Bases distribuidas}

Al distribuir una base de datos, a la independencia entre datos físicos y lógicos se agregan:

\begin{description}
	\item[Organización de los datos:] no debe ser necesario conocer detalles de la red ni la ubicación de los datos. Tiene dos subcasos:
	\begin{description}
		\item[Transaparencia de ubicación:] independencia entre ubicación de los datos y dónde se lanza la consulta
		\item[Transparencia de nombre:] con saber el nombre del objeto a referenciar debe ser suficiente, no hace falta saber su ubicación
	\end{description}
	\item[Fragmentación:] no debe hacer falta conocer la existencia de fragmentos
	\begin{description}
		\item[Horizontal (sharding):] distribuye la tabla en subconjuntos de filas
		\item[Vertical:] distribuye la tabla en subconjuntos de columnas
	\end{description}
	\item[Replicación:] almacenar datos en varios lugares para mejorar disponibilidad, performance, confiabilidad. El usuario no debe necesitar la conocer la existencia de copias
\end{description}

Propiedades deseables:

\begin{description}
	\item[Disponibilidad/Confiabilidad:] probabilidad de que un sistema se encuentre operando (son conceptos diferentes pero muy similares)
	\item[Escalabilidad:] medida en que un sistema puede expandir su capacidad mientras continúa operando. Puede ser:
	\begin{description}
		\item[Horizontal:] expandir la cantidad de nodos
		\item[Vertical:] expandir la capacidad de un nodo individual
	\end{description}
	\item[Tolerancia a partición:] que el sistema siga funcionando a pesar de que la red se particione
	\item[Autonomía:] en qué medida nodos individuales pueden operar independientemente
\end{description}

\subsection{Control de Concurrencia}

Si un item se encuentra replicado en varios nodos, debe controlarse para mantener la consistencia.

\begin{description}
	\item[Copia distinguida:] se designa una copia de cada data item como copia distinguida. Pedidos de lock y unlock enviados al sitio de la copia distinguida. Existen 3 variantes:
	\begin{description}
		\item[Sitio primario:] todas las copias primarias se encuentran en el mismo sitio. 2PC garantiza seriabilidad. Genera un cuello de botella y punto único de falla
		\item[Sitio primario + Backup:] hay un segundo nodo que actúa como backup. La información de los locks se mantiene en ambos sitios, disminuyendo la performance. Si se cae el sitio primario, el backup toma el control
		\item[Copia primaria:] admite copias distinguidad en diferentes sitios. Elimina el cuello de botella, y si un sitio falla, solo se ven afectadas las transacciones que necesitan un lock de ese sitio. Puede sumarse backup
	\end{description}
	\item[Votación:] se pide lock a todos los sitios con una copia del item. Se consigue el lock si la mayoría de sitios dan el OK. Implica mayor cantidad de mensajes y se complejiza teniendo en cuenta que algunos sitios pueden caerse durante la votación
\end{description}

2PC vs 3PC

Sistemas de replicacion:

master/slave

peer to peer

Homogéneo: todos los nodos corren el mismo software (o compatible), y comparten esquemas.

Heterogéneo: cada nodo usa software y esquema distinto y colaboran de forma limitada.
